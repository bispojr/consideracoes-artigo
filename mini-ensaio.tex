\documentclass[12pt]{article}

\usepackage{sbc-template}

\usepackage{graphicx,url}

\usepackage[brazil]{babel}   
%\usepackage[latin1]{inputenc}  
\usepackage[utf8]{inputenc} 
\usepackage{hyperref} 
% UTF-8 encoding is recommended by ShareLaTex

     
\sloppy

\title{Considerações sobre o Artigo ``Realidade escolar e formação docente em cursos de licenciatura''}

\author{Esdras L. Bispo Jr.\inst{1,2}}


\address{Instituto de Ciências Exatas e Tecnológicas (ICET)
  (ICET)\\
  Universidade Federal de Jataí (UFJ)
\nextinstitute
  Jataí ACM SIGCSE {\it Chapter}\\
  Grupo de Interesse Especial em Educação de Computação
  \email{bispojr@ufg.br}
}

\begin{document} 

\maketitle


O artigo\footnote{Disponível em \url{www.unesp.br/aci/debate/180210-rosemaraperpetualopes.php}.}, escrito pela Profa. Rosemara Lopes, traz algumas inquietações sobre a vivência dos licenciandos em relação ao estágio curricular obrigatório. Algumas destas questões são referentes à frustração dos discentes ao se deparar com a realidade do cenário da educação básica. Outras questões são concernentes ao próprio sistema de formação ao qual eles estão inseridos.

Pelo que se compreende do texto, a provocação maior levantada pela autora é esta
\begin{quote}
\sc ``Não poderiam os estágios, tendo em vista o seu papel na formação do professor, ser mais bem aproveitados nas licenciaturas dos cursos presenciais [...] ?''	
\end{quote}

A primeira ponderação a ser feita, a partir de elementos mencionados pelo próprio texto, diz respeito ao início deste contato com a realidade da educação básica. A respeito do fato de que apenas no final do curso que esta vivência é efetivamente concretizada (embora que a previsão seja para iniciar desde o início do curso), a autora afirma:
\begin{quote}
	\sc
``Entretanto, nos primeiros semestres, os alunos não vão para a sala de aula vivenciar a profissão, até porque não dispõem de conhecimentos que os capacitem para tal logo no início do curso.''
\end{quote}
A bem da verdade, seria bastante interessante que este contato fosse feito inicialmente, mesmos sem tais conhecimentos. Há uma ligeira impressão de que tais conhecimentos só seriam obtidos dentro do espaço de um curso de licenciatura. Se admite-se o pressuposto de que a abordagem construtivista parte da realidade do próprio educando, em que este molda e adapta o próprio modelo da realidade por ele construído, a proposta do estágio apresentada pode estar sendo desvinculada desta hipótese. O processo de construção, desconstrução e reconstrução do licenciado inicia a partir de sua própria realidade, com os conhecimentos que estes já carregam consigo mesmo, independente das novas perspectivas que o discente será confrontado durante o seu processo de formação.

A segunda ponderação - a inovação não é ensinada, só a reprodução.
\begin{quote}
	\sc
``Como pode um professor inovar em sua prática utilizando o laboratório de Informática da escola, se desconhece as possibilidades de uso do computador para fins educativos?''
\end{quote}

\begin{quote}
	\sc
``Por que não preparar os futuros professores para atuar na escola da sociedade contemporânea [...] ?''
\end{quote}

A terceira ponderação - a reprodução leva a uma ausência de reflexão da própria prática docente.
\begin{quote}
	\sc
``Tudo em nome de uma pretensa modernização que, por si só, não implica em qualidade de ensino. ''
\end{quote}


\bibliographystyle{sbc}
%\bibliography{sbc-template}

\end{document}
